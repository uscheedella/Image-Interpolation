\begin{enumerate}
    \item \texttt{matrixTest}
    \begin{itemize}
        \item Use the \texttt{listings} package to include your \texttt{matrixTest} output in the pdf.
        \item For each bug, use the \texttt{listings} package to display the original line of code with the error, as well as the fix.  Describe the error.            
    \end{itemize}
    \textbf{ANSWER:} % answer goes here

    \lstinputlisting[language={},frame=single,title=gdbOutput.txt]{gdbOutput.txt}
\newpage 

\begin{lstlisting}[frame=single]
--- a/labs/lab04_sol/code/gdb/matrixProd.c
+++ b/labs/lab04_sol/code/gdb/matrixProd.c
@@ -7,7 +7,7 @@ double dotProd(int n, double* x, double* y){
 
   double dotProd = 0;
 
-  for(int i=0; i<=n; i++){
+  for(int i=0; i<n; i++){
     dotProd += x[i]*y[i];
   }
\end{lstlisting}
The loop counts 1 extra time, from 0 to $N$, when it should count from 0 to $N-1$.


\begin{lstlisting}[frame=single]
--- a/labs/lab04_sol/code/gdb/matrixProd.c
+++ b/labs/lab04_sol/code/gdb/matrixProd.c
@@ -22,7 +22,7 @@ double dotProd(int n, double* x, double* y){
 void matrixVecProd(int m, int n, double* A, double* x, double* b){
 
   for(int i=0; i<m; i++){
-    b[i] = dotProd(n, A+i*m, x);
+    b[i] = dotProd(n, A+i*n, x);
   }
 }
\end{lstlisting}
The calculation of the start of the row in the mat-vec is not correct.  It assumes rows are length $M$ and not $N$.

\begin{lstlisting}[frame=single]
--- a/labs/lab04_sol/code/gdb/matrixTest.c
+++ b/labs/lab04_sol/code/gdb/matrixTest.c
@@ -118,7 +118,7 @@ void testShortFat(){
   double ans[2] = {14, 32};
 
   
-  double* c = vecDiff(n, b, ans);
+  double* c = vecDiff(m, b, ans);

   if(norm(m, c)<tol){ 
     printf("Test passed!\n");
\end{lstlisting}
For this test case, the vector b has length $M$, not $N$.
   
\begin{lstlisting}[frame=single]
--- a/labs/lab04_sol/code/gdb/matrixTest.c
+++ b/labs/lab04_sol/code/gdb/matrixTest.c
@@ -155,7 +155,7 @@ void testTallSkinny(){
     }
   }
 
-  printf("Testing matrixVecProd a short and fat matrix\n");
+  printf("Testing matrixVecProd a tall and skinny matrix\n");
 
   printA(m, n, A);
\end{lstlisting}
The print text refers to the wrong test.
 
\begin{lstlisting}[frame=single]
--- a/labs/lab04_sol/code/gdb/matrixTest.c
+++ b/labs/lab04_sol/code/gdb/matrixTest.c
@@ -170,7 +170,7 @@ void testTallSkinny(){
   
   double* c = vecDiff(m, b, ans);
   
-  if(norm(m, b)<tol){
+  if(norm(m, c)<tol){
     printf("Test passed!\n");
   }
   else{
\end{lstlisting}
The test should check the norm of the error vector, c, not the right-hand side vector b.
    
    \newpage
    \item \texttt{fibonacci}
    \begin{itemize}
        \item Use the \texttt{listings} package to include your \texttt{fibonacci} output in the pdf.
        \item For each bug, use the \texttt{listings} package to display the original line of code with the error, as well as the fix.  Describe the error.
    \end{itemize}
    \textbf{ANSWER:} % answer goes here

    \lstinputlisting[language={},frame=single,title=fibOutput.txt]{fibOutput.txt}

\begin{lstlisting}[frame=single]
--- a/labs/lab04_sol/code/valgrind/fibonacci.c
+++ b/labs/lab04_sol/code/valgrind/fibonacci.c
@@ -4,7 +4,7 @@
 int main (int argc, char **argv) {
     //initialize variables
     int array_size = 30;
-    int *nums = (int *) malloc(array_size);
+    int *nums = (int *) malloc(array_size*sizeof(int));
     
     //seed with first two values
     nums[0] = 1;
\end{lstlisting}

The allocation if the \texttt{nums} array is missing the \texttt{sizeof}, so an incorrect number of bytes is allocated.

    \newpage
    \item \texttt{pascal}
    \begin{itemize}
        \item Use the \texttt{listings} package to include your \texttt{pascal} output in the pdf.
        \item For each bug, use the \texttt{listings} package to display the original line of code with the error, as well as the fix.  Describe the error.
    \end{itemize}
    \textbf{ANSWER:} % answer goes here

    \begin{tiny}
    \lstinputlisting[language={},frame=single,title=pasOutput.txt]{pasOutput.txt}
    \end{tiny}
    
\begin{lstlisting}[frame=single]
--- a/labs/lab04_sol/code/valgrind/pascal.c
+++ b/labs/lab04_sol/code/valgrind/pascal.c
@@ -7,7 +7,7 @@ int main(int argc,char **argv) {
     char depth = 17;
     char spacing = 5;
     char spacing_start = 3;
-    char length = depth*depth;
+    int length = depth*depth;
     
     //initial setup
     int *nums = (int *) malloc(length*sizeof(int));

\end{lstlisting}

The maximum number that a char can store is 255, but \texttt{depth*depth} is 289, so there is an overflow.  We fix by using a number that can store more values.

    \newpage
    \item \texttt{array\_sum}
    \begin{itemize}
        \item Use the \texttt{listings} package to include your \texttt{array\_sum} output in the pdf.
        \item For each bug, use the \texttt{listings} package to display the original line of code with the error, as well as the fix.  Describe the error.
    \end{itemize}
    \textbf{ANSWER:} % answer goes here

    \lstinputlisting[language={},frame=single,title=sumOutput.txt]{sumOutput.txt}

    
\begin{lstlisting}[frame=single]
--- a/labs/lab04_sol/code/valgrind/array_sum.c
+++ b/labs/lab04_sol/code/valgrind/array_sum.c
@@ -6,7 +6,7 @@ int main(int argc, char **argv) {
     int arr_length = 5*5;
 
     float *sum = (float *) malloc(sum_length*sizeof(float));
-    float *arr = (float *) malloc(sum_length*sizeof(float));
+    float *arr = (float *) malloc(arr_length*sizeof(float));
 
     //seed array with random numbers
     srand48(0);

\end{lstlisting}

The \texttt{arr} array was not allocated with the correct length so the reads in constructing the sum access out of bounds.

    \newpage
    \item \texttt{rotate\_vector}
    \begin{itemize}
        \item Use the \texttt{listings} package to include your \texttt{rotate\_vector} output in the pdf.
        \item For each bug, use the \texttt{listings} package to display the original line of code with the error, as well as the fix.  Describe the error.
    \end{itemize}
    \textbf{ANSWER:} % answer goes here

    \lstinputlisting[language={},frame=single,title=rotOutput.txt]{rotOutput.txt}
    
\begin{lstlisting}[frame=single]
--- a/labs/lab04_sol/code/valgrind/rotate_vector.c
+++ b/labs/lab04_sol/code/valgrind/rotate_vector.c
@@ -14,7 +14,7 @@ int main(int argc, char **argv) {
 
     //fill rows of array with vector rotations
     for (int i = 0; i < vector_size; i++) {
-       for (int j = 0; j < vector_size; i++) {
+       for (int j = 0; j < vector_size; j++) {
            int index = (j+i)%vector_size;
            rotations[i*vector_size + index] = vector[j];
        }
\end{lstlisting}
    
The loop increment uses the wrong variable, so \texttt{j} does not change and \texttt{i} increments faster than it should.

\end{enumerate}
