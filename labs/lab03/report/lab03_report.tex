\documentclass[letter]{article}

\usepackage[margin=1in]{geometry} % make more efficient use of the page

\usepackage[utf8]{inputenc}

\usepackage{amsmath} % math tools
\usepackage{amsfonts} % math tools
\RequirePackage{amssymb} % math tools
\RequirePackage{amsbsy} % math tools

\renewcommand{\vec}[1]{\ensuremath{\boldsymbol{#1}}} % make vectors nicer

\usepackage{graphicx} % graphics
\usepackage{xcolor} % colored text

\usepackage{hyperref} % URLs and such
\usepackage{verbatim} % allows \verb-- command

\usepackage{algorithmicx} % algorithm environment
\usepackage{listings} % code listings

\usepackage{natbib} % bibliography


\title{CMDA 3634 \\ Lab 03 Report}
\author{Your Name Here}
% \date{}


\begin{document}

\maketitle

\section*{Part A}

\begin{enumerate}
    \item Use the \texttt{listings} package to include your output (\texttt{output\_pt\_a.txt}) in your pdf.  You will need to copy \texttt{output\_pt\_a.txt} to the reports directory.
    
    \textbf{ANSWER:} % answer goes here

    % Hint: the correct latex command is \lstinputlisting[language={}]{output.txt}

    \item The \texttt{axpy} routine no longer uses the return value, as the return variable is an argument to the function.  How can we make use of the function return value to get some use out of it?

    \textbf{ANSWER:} % answer goes here

    \item For these routines, give one reason why we might choose to pass only the structures by pointer and not the scalar values?
    
    \textbf{ANSWER:} % answer goes here

    \item Other than the instructor or TAs, who did you receive assistance from on this assignment?
    
    \textbf{ANSWER:} % answer goes here
\end{enumerate}

% \section*{Part B}

% \begin{enumerate}
    \item Use the \texttt{listings} package to include your output (\texttt{output\_pt\_b.txt}) in your pdf.  You will need to copy \texttt{output\_pt\_b.txt} to the reports directory.

    \textbf{ANSWER:} % answer goes here
	\lstinputlisting[language={}]{output_pt_b.txt}
    % Hint: the correct latex command is \lstinputlisting[language={}]{output.txt}

    \item Call your \texttt{sum} function with a value for \texttt{N} that is larger than the number of entries in the array.  What happens when you compile?  When you run?  What is happening here?

    \textbf{ANSWER:} This compiles with no issue.  There is no compile-time check for array out of bounds.  When we run there is a segmentation fault for attempting to access a stack array out of bounds, triggered by the system detecting the out of bounds access attempt.  This is a system security mechanism.

    \item Other than the instructor or TAs, who did you receive assistance from on this assignment?
    
    \textbf{ANSWER:} No one.
\end{enumerate}




% I leave this part commented out as some example code if you need it.
% \section{Introduction}
% There is a theory which states that if ever anyone discovers exactly what the Universe is for and why it is here, it will instantly disappear and be replaced by something even more bizarre and inexplicable.
% There is another theory which states that this has already happened.

% \begin{figure}[h!]
% \centering
% \includegraphics[scale=1.7]{universe}
% \caption{The Universe}
% \label{fig:universe}
% \end{figure}

% \section{Conclusion}
% ``I always thought something was fundamentally wrong with the universe'' \citep{adams1995hitchhiker}

% \bibliographystyle{plain}
% \bibliography{references}
\end{document}
