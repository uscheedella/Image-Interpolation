\begin{enumerate}
    \item Use the \texttt{listings} package to include your output (\texttt{output\_pt\_a\_vector.txt}) in your pdf.  You will need to copy \texttt{output\_pt\_a\_vector.txt} to the reports directory.
    % Hint: the correct latex command is \lstinputlisting[language={}]{output.txt}

    \textbf{ANSWER:} % answer goes here
    

    \item For each of the following use-cases, indicate if the specified array should be allocated on the stack, the heap, or either.  Explain your selection.
    \begin{enumerate}
        \item An array of integers length 10 in a function that is called a small number of times.
        \item An array of doubles of length 3, where $\sim 10^3$ instances exist and frequently used in the program.
        \item An array of doubles of length 3, where $\sim 10^4$ instances exist and frequently used in the program.
        \item An array of doubles of length 3, where $\sim 10^5$ instances exist and frequently used in the program.
        \item An array of doubles of length 3, where $\sim 10^6$ instances exist and frequently used in the program.
        \item An array of doubles of length 3, where $\sim 10^8$ instances exist and frequently used in the program.
        \item An array of floats of length 10,000, to be used throughout the whole program.
        \item An array of floats of length 10,000, to be used in a single function.
    \end{enumerate}

    \textbf{ANSWER:} % answer goes here
    

    \item In C, there is no mechanism to see if a pointer points to heap memory that has already been allocated, so we cannot be sure that we do not re-allocate an array.  How can we code defensively to ensure that this does not happen?

    \textbf{ANSWER:} % answer goes here
    

    \item Other than the instructor or TAs, who did you receive assistance from on this assignment?
    
    \textbf{ANSWER:} % answer goes here
    
\end{enumerate}
